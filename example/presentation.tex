\documentclass{beamer}
\usetheme{yale}

% Use the official Yale fonts with XeLaTeX
\usepackage{mathspec}
\setmainfont[Numbers=OldStyle]{YaleNew}
\setsansfont{Mallory}
\setmathfont{Computer Modern}
\usefonttheme{professionalfonts}

% Use Computer Modern Unicode for IPA symbols
\newfontfamily{\ipafont}{cmunrm.otf}

% Automatically make slides that introduce sections
\AtBeginSection{
\begin{frame}
	\begin{center}
		\structure{\Huge \insertsection}
	\end{center}
\end{frame}
}

% Text
\usepackage{ulem}
\usepackage{xltxtra}

% Math
\usepackage{amsthm}
\usepackage{mathrsfs}
\allowdisplaybreaks

% Theorem styles
\newtheorem{proposition}[equation]{Proposition}

\theoremstyle{definition}
\newtheorem{remark}[equation]{Remark}

\theoremstyle{example}
\newtheorem{longproof}[equation]{Proof}

% Linguistics
\usepackage[notipa]{ot-tableau}
\usepackage{gb4e}
\noautomath

% Pictures
\usepackage{tikz}
\usepackage{tikz-qtree}
\usepackage{tikz-qtree-compat}
\tikzset{every tree node/.style={align=center,anchor=north}}

% Information about your presentation
\title{Yale Beamer Template}
\subtitle{Sample Presentation}
\author{Yiding Hao}
\institute{\textsc{department of linguistics $\cdot$ yale university}}
\date{\today}
\venue{\textsc{beamer template}}

\begin{document}
	
% Display the information in the heading
\frame{
	\titlepage
}
	
\begin{frame}{Outline}
	Use \texttt{\textbackslash tableofcontents} to include a table of contents.
	\tableofcontents
\end{frame}

\section{Basic Beamer Functionality}

\begin{frame}{Slide Content, Lists, Etc.}

	Here's what a list looks like.\pause
	\begin{itemize}[<+->]
		\item You can
		\item make items appear
		\item one at a time.
	\end{itemize}

	\begin{center}
		\only<5>{\textit{Use standard} \texttt{beamer} \textit{functionality to implement timing!}}
	\end{center}
	
\end{frame}



\section{Mathematics}

\begin{frame}{Equations}

	Add equations to your slides as usual.
	\[
	f(a) = \frac{1}{2\pi i} \oint_\gamma \frac{f(z)}{z - a} \, dz
	\]
	
	Create multi-line equations using \texttt{align}.
	\begin{align*}
		\int_a^b x^2 \, dx &= \frac{x^3}{3} \Big|_a^b \\
		&= \frac{b^3 - a^3}{3}
	\end{align*}


\end{frame}

\begin{frame}{Theorems and Proofs}

\begin{proposition}[Bayes's Theorem]
	Let $A$ and $B$ be random variables. Then,
	\[
	P(A|B) = \frac{P(B|A)P(A)}{P(B)}.
	\]
\end{proposition} 
\begin{proof}
	\[
	P(A|B) = \frac{P(A\cap B)}{P(B)} = \frac{P(A\cap B)P(A)}{P(B)P(A)} = \frac{P(B|A)P(A)}{P(B)}.
	\]
\end{proof}

\end{frame}

\begin{frame}{Definitions and Examples}

	\begin{definition}
		A \textit{function} is a set $f$ of ordered pairs such that if $\langle x, y \rangle \in f$ and $\langle x, z \rangle \in f$, then $y = z$.
	\end{definition}

	\begin{example}
		Suppose a pizza has radius $z$ and thickness $a$. Then, its volume is
		\[
		\pi z^2 a = pizza.
		\]
	\end{example}

\end{frame}



\section{Linguistics}

\begin{frame}{Numbered Examples}
	It looks like \texttt{gb4e} uses roman type by default for the glosses, and sans-serif for the translation.
	\begin{exe}
		\ex {\gll Ni-na-enda maktaba-ni kwa ajili ya ku-tum-ia choo.\\
				\textsc{1sg}-\textsc{pres}-go library-\textsc{loc} for purpose of \textsc{inf}-use-\textsc{appl} restroom \\
			\trans ``I'm going to the library to use the restroom.''}
	\end{exe}
\end{frame}

\begin{frame}{Syntactic Trees}
	\begin{exe}
		\ex \dots vitsi ras \sout{Rezo ch'ams} ``\dots know what \sout{Rezo eats}''\\
		\begin{tikzpicture}[scale=0.9]
		\Tree [.VP {V\\vitsi} [.CP {C\\{\footnotesize $\begin{bmatrix}
				\textbf{Q} & +\\
				\end{bmatrix}$}} [.FocP \node(op){DP\\ras\\{\footnotesize $\begin{bmatrix}
				\textbf{wh} & +\\
				\end{bmatrix}$}}; [.Foc$^\prime$ {Foc\\{\footnotesize $\begin{bmatrix}
				\textit{wh} & +\\
				\text{EPP} & \\
				\text{E} &
				\end{bmatrix}$}} [.{\sout{TP}} \edge[roof]; \node(t){\sout{Rezo ch'ams $t$}}; ] ] ] ] ]
		
		\begin{pgfinterruptboundingbox}
		\draw [->] (t) [in=-90,out=250,looseness=1] to (op);
		\end{pgfinterruptboundingbox}
		\node[below of=t] {};
		\end{tikzpicture}
	\end{exe}
\end{frame}

\begin{frame}{Optimality Theory}
	If you use {\ipafont \XeLaTeX}, you can just type IPA directly into the \texttt{.tex} source.
	
	\begin{center}
		\begin{tableau}{c|c|c|c|c}
			\inp{\ipafont fʕal} \const{*Complex-Onset} \const{Max} \const{Id} \const{Onset} \const{Dep}
			\cand{\ipafont fʕal} \vio{*!} \vio{} \vio{} \vio{} \vio{}
			\cand{\ipafont ʕal} \vio{} \vio{*!} \vio{} \vio{} \vio{}
			\cand{\ipafont iʕal} \vio{} \vio{} \vio{*!} \vio{*} \vio{}
			\cand[\Optimal]{\ipafont ifʕal} \vio{} \vio{} \vio{} \vio{*} \vio{*}
		\end{tableau}
	\end{center}
\end{frame}


\end{document}

